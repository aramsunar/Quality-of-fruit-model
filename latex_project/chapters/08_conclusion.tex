\chapter{CONCLUSION}

This research project successfully developed and evaluated convolutional neural network architectures for automated fruit quality assessment, systematically exploring how input data characteristics and learning paradigms influence classification performance. Through controlled experimental scenarios spanning single-task and multi-task learning frameworks, we addressed fundamental questions about the necessity of colour information, the efficacy of data augmentation, and the potential benefits of multi-objective optimisation.

\section{Key Findings from Part A: Single-Task Learning}

The single-task quality classification experiments established strong baseline performance while revealing surprising insights about feature dependencies and augmentation effects.

\textbf{Baseline RGB Performance (Scenario 1):} The SimpleCNN architecture achieved exceptional 99.84\% validation and 99.79\% test accuracy on standard three-channel colour images, demonstrating that relatively simple convolutional networks can effectively capture discriminative quality features. Perfect AUC scores (1.0000) across all quality classes confirmed excellent probability calibration and rank-ordering capability. Error analysis revealed that misclassifications occurred exclusively at adjacent quality boundaries (particularly Mild-Rotten transitions), indicating genuine ambiguity in borderline cases rather than systematic model failures.

\textbf{Grayscale Equivalence (Scenario 2):} Contrary to intuitive expectations, grayscale conversion preserved nearly identical performance (99.84\% validation, 100\% test), challenging assumptions about colour's necessity for fruit quality assessment. This remarkable result suggests that texture patterns, shape characteristics, and intensity gradients captured in single-channel representations contain sufficient discriminative information for this task. From a practical standpoint, these findings validate grayscale-based deployment scenarios that offer 66\% input dimensionality reduction without sacrificing classification accuracy.

\textbf{Augmentation Degradation (Scenario 3):} Data augmentation unexpectedly reduced performance (98.99\% validation, 99.25\% test) despite theoretical expectations of improved generalisation. Analysis attributed this decline to ceiling effects where baseline performance left minimal improvement room, aggressive photometric transformations that genuinely altered perceived quality categories, and slower convergence dynamics requiring 3-4× more training epochs. The augmented model exhibited persistent training volatility and struggled to surpass 98-99\% accuracy, suggesting that the comprehensive augmentation pipeline (geometric + photometric) proved too aggressive for this classification task where subtle quality indicators require preservation.

\section{Methodological Contributions}

This research established a systematic evaluation framework combining multiple analytical perspectives:

\begin{itemize}
    \item \textbf{Multi-metric assessment}: Accuracy, precision, recall, F1-score, and AUC provide complementary performance views
    \item \textbf{Confusion matrix analysis}: Reveals class-specific error patterns and systematic biases
    \item \textbf{ROC curve examination}: Evaluates discriminative capability across decision thresholds
    \item \textbf{Training dynamics investigation}: Monitors convergence stability, overfitting indicators, and learning rate scheduling effects
\end{itemize}

This comprehensive approach ensures robust evaluation beyond single-metric comparisons, building confidence in model reliability for practical deployment.

\section{Practical Implications}

The experimental findings yield several actionable insights for fruit quality assessment system deployment:

\begin{enumerate}
    \item \textbf{Grayscale viability}: Grayscale-based systems offer computational efficiency (reduced bandwidth, storage, processing) without sacrificing accuracy, particularly valuable for resource-constrained embedded deployments or large-scale industrial applications

    \item \textbf{Augmentation restraint}: Aggressive augmentation proves counterproductive when baseline performance already approaches optimality; gentler transformations or task-specific augmentation policies that preserve subtle quality indicators may prove more appropriate

    \item \textbf{Architecture simplicity}: The SimpleCNN architecture with four convolutional blocks and three fully connected layers achieved near-perfect performance, suggesting that extremely deep networks may be unnecessary for this classification task

    \item \textbf{Error patterns}: Misclassifications concentrate at adjacent quality boundaries (Good-Mild, Mild-Rotten) rather than quality extremes (Good-Rotten), indicating that improved discrimination of transitional states represents the primary remaining challenge
\end{enumerate}

\section{Multi-Task Learning Perspective (Part B)}

Part B extends this analysis to multi-task learning frameworks where networks simultaneously predict fruit type and quality level. The multi-task paradigm addresses efficient data utilisation by leveraging dual-label information already present in dataset organisation, while testing whether shared convolutional backbones can serve multiple related objectives without task interference.

Early findings from multi-task scenarios suggest that:
\begin{itemize}
    \item Fruit type identification achieved perfect 100\% accuracy across all scenarios, indicating highly distinctive inter-fruit visual characteristics
    \item Quality classification performance in multi-task models closely matched or marginally exceeded single-task baselines, providing evidence for positive transfer learning effects
    \item Shared feature representations proved sufficient for both objectives without requiring substantially increased parameter counts
\end{itemize}

\section{Limitations and Future Directions}

Several limitations warrant acknowledgment and suggest directions for future investigation:

\textbf{Dataset Specificity:} The FruQ-multi dataset contains high-quality images captured under controlled conditions with consistent lighting and backgrounds. Real-world deployment scenarios with variable imaging conditions, occlusions, or multiple fruits per image may present additional challenges not addressed in this controlled evaluation.

\textbf{Class Imbalance:} Substantial imbalance exists both between fruit types (tomatoes: 1990 images vs. strawberries: 216 images) and within quality categories for specific fruits. While class weighting addressed this partially, extreme imbalance may still affect model learning, particularly for underrepresented categories.

\textbf{Augmentation Exploration:} The comprehensive augmentation pipeline proved too aggressive, but systematic exploration of gentler transformation ranges, selective augmentation application (geometric-only or photometric-only), or quality-class-specific augmentation policies might yield improved results.

\textbf{Architecture Search:} While SimpleCNN achieved excellent performance, deeper architectures (DeepCNN, LightCNN) or modern designs incorporating residual connections, attention mechanisms, or squeeze-and-excitation blocks might offer incremental improvements or better parameter efficiency.

\textbf{Temporal Quality Dynamics:} This research treats quality assessment as static classification, but fruit degradation follows temporal trajectories. Future work could explore ordinal regression approaches that model quality progression or temporal sequence models that predict degradation rates.

\section{Concluding Remarks}

This research demonstrates that convolutional neural networks provide highly effective automated fruit quality assessment, achieving near-perfect classification accuracy (>99.8\%) while revealing surprising insights about feature dependencies. The remarkable performance of grayscale-based models challenges conventional assumptions about colour's necessity, offering practical deployment advantages without accuracy sacrifices. Conversely, data augmentation's performance degradation highlights the importance of task-appropriate preprocessing strategies, particularly when baseline performance already approaches optimality.

The systematic experimental design, controlling for individual variables across scenarios, provides reliable evidence for architectural and preprocessing decisions in fruit quality assessment systems. The comprehensive evaluation methodology, incorporating multiple metrics, visualisations, and training dynamics analysis, establishes a rigorous framework for future agricultural image classification research.

Looking forward, the integration of multi-task learning paradigms (Part B) with efficient grayscale processing promises practical systems that simultaneously identify fruit types and assess quality with minimal computational overhead, potentially enabling real-time classification in industrial agricultural supply chains. The foundation established through this controlled experimental investigation provides a solid basis for such deployments while highlighting remaining challenges in borderline quality discrimination that represent opportunities for continued refinement.
